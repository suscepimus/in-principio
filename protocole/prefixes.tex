% !TEX TS-program = LuaLaTex+se

\documentclass[a5paper]{article}
\usepackage[frenchb]{babel}
\usepackage{fullpage}
\usepackage{gregoriotex}
\usepackage{libertine}
\usepackage{multirow}
\usepackage{longtable}
\pagestyle{empty}
\usepackage{microtype}
\usepackage{realscripts}

\begin{document}
\renewcommand{\arraystretch}{1.25}

\begin{center}
\begin{Large}
Protocole de nommage
\end{Large}
\end{center}

{\itshape On trouvera ici les règles de nommage des fichiers `gabc' appliquées à Solesmes.
 Le principe est de pouvoir identifier la pièce assez facilement par son incipit, 
 précédé du genre de pièce, et suivi éventuellement de toute précision utile pour 
 la différencier de pièces proches par le nom.}

\begin{longtable}{r|l}
\textbf{Abréviation} & \textbf{Nature de la pièce} \\
\hline	\endhead
 \multicolumn{2}{r}{\textit{\footnotesize TSVP→}}\endfoot
 \endlastfoot
AA & Antienne alléluiatique				\\
AG & Agnus Dei						 \\
AL & Verset d'alléluia						 \\
AN & Antienne						 \\
AP & Antienne avec psaume					\\
BS & Bénédiction solennelle				\\
CA & Cantique						 \\
CO & Antienne de communion						 \\
CR & Credo						 \\
GL & Gloria						 \\
GR & Répons Graduel						 \\
HF & Hymne intégrale						 \\
HY & Hymne 1{\ier} verset seulement						 \\
IN & Antienne d'Introït						 \\
IP & Intonation de Psaume						 \\
KY & Kyrie						 \\
LL & Lecture						 \\
LT & Litanie						 \\
OD & Ordo divers						 \\
OF & Antienne d'offertoire						 \\
OS & Oraison sur le ton solennel				\\
OX & Oraison	sur le ton simple					 \\
PE & Prière eucharistique						 \\
PF & Préface sur le ton solennel						 \\
PR & Prose						 \\
PS & Psaume						 \\
PX & Préface sur le ton simple						 \\
RA & Répons bref alléluiatique						 \\
RB & Répons bref						 \\
RN & Répons neumé						 \\
RP & Répons prolixe						 \\
SA & Sanctus						 \\
SQ & Séquence						 \\
TR & Trait						 \\
VR & Verset de répons			\\
VS & Versus

%\begin{tabular}{|*{4}{c|}}

\end{longtable}

\end{document}

