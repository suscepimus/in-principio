% !TEX TS-program = LuaLaTex+se

\documentclass[11pt, a4paper]{article}
\usepackage{fullpage}
\usepackage{libertine}
\usepackage{gregoriotex}
\usepackage{multirow}
\usepackage{longtable}

\begin{document}
\xdef\grefinalpenalty{0}%
\gresetlines{invisible}

\gresetclef{invisible}

%\def\tld{\raisebox{-.8ex}{\~~}\hspace{-0.15em}}

\def\tld{\gretilde}

\newcommand{\gs}[1]{\tt \noindent \gabcsnippet{#1}}

\gresetinitiallines0

\gresetlyriccentering{firstletter}

\grechangedim{interglyphspace}{0.05927 cm}{scalable}%

\grechangedim{glyphspace}{0.21877 cm}{scalable}%

\grechangedim{spaceafterlineclef}{0 cm}{scalable}%

\grechangedim{bitrivirspace}{0.05927 cm}{scalable}%

\grechangedim{bitristrospace}{0.03 cm}{scalable}%

\begin{center}
\begin{Large}
TABULA NEUMARUM\\
IN TYPIS SOLESMENSIBUS SERVATIS
\end{Large}
\end{center}

\begin{longtable}{|c|c|c|c|}

%\begin{tabular}{|*{4}{c|}}
\hline
Neumæ			&\multicolumn{3}{c|}{Exempla Figurarum}							\\
\cline{2-4}
 aut				&	 				&\multicolumn{2}{c|}{Figuræ liquescentes}		\\
\cline{3-4}
Neumarum elementa	&		Figuræ		&	Figuræ 		&	Figuræ			\\
				&		rectæ		&		auctæ	& deminutæ			\\
\hline
\endfirsthead
Neumæ		 	&Figuræ rectæ		&	Figuræ auctæ		&	Figuræ deminutæ	\\
\hline
\endhead
Punctum			& \gs{g(g)}		& \gs{g>(g>) g<(g<)}	& \gs{g<v>\tld</v>(g~)}	\\
\hline
Punctum inclinatum	&\gs{G(G)}		&\gs{G>(G>)}			&\gs{G<v>\tld</v>(G~)}	\\
\hline
Virga				&\gs{gv(gv)}		&					&					\\
\hline
Bivirga vel trivirga	&\gs{gvv(gvv) gvvv(gvvv)}&				&					\\
\hline
Virga	reversa		&\gs{gV(gV)}		&					&					\\
\hline
Quilisma			&\gs{gw(gw)}		&					&					\\
\hline
Apostropha vel stropha
				&\gs{gs(gs)}		&\gs{gs>(gs>)}			&					\\
\hline
Distropha vel tristropha
				&\gs{gss(gss) gsss(gsss)}
								&\gs{gss>(gss>) gsss>(gsss>)}
													&					\\
\hline
Oriscus			&\gs{go(go> go<)}	&					&\gs{go<v>\tld</v>(go~)}	\\
\hline
Podatus			&\gs{fh(fh)}		&\gs{fh>(fh>) fh<(fh<)}	&\gs{fh<v>\tld</v>(fh~)}	\\
% \hline
% Podatus initio debilis	&\gs{-fh(-fh)}		&\gs{-fh>(-fh>) -fh<(-fh<)}
% 													&\gs{-fh<v>\tld</v>(-fh~)}\\
\hline
Clivis				&\gs{hf(hf)}		&\gs{hf>(hf>) hf<(hf<)}	&\gs{hf<v>\tld</v>(hf~)}	\\
\hline
Pes quassus		&\gs{foh(foh)}		&\gs{foh>(foh>) foh<(foh<)}
													&\gs{foh<v>\tld</v>(foh~)}\\
\hline
Quilisma pes		&\gs{gwh(gwh) gWi(gWi)}
								&\gs{gwh>(gwh>) gwh<(gwh<)}
													&\gs{gwh<v>\tld</v>(gwh~)}\\
\hline
Torculus vel pes flexus
				&\gs{ghg(ghg) fig(fig)}
								&\gs{fig>(fig>)}			&\gs{fhe<v>\tld</v>(fhe~)}\\
% \hline
% Torculus initio debilis	&\gs{-ghg(-ghg)}	&\gs{-fig>(-fig>)}		&\gs{-fhe<v>\tld</v>(-fhe~)}\\
\hline
Torculus resupinus	&\gs{fgeg(fgeg)}	&\gs{fgeg<(fgeg<)}		&\gs{fgeg<v>\tld</v>(fgeg~)}\\
\hline
Porrectus vel flexa resupina
				&\gs{geh(geh)}		&\gs{geh>(geh>)}		&\gs{geh<v>\tld</v>(geh~)}\\
\hline
Porrectus flexus		&\gs{gehf(gehf)}	&\gs{gehf>(gehf>)}		&\gs{gehf<v>\tld</v>(gehf~)}\\
% \hline
% Porrectus initio debilis&\gs{-fgeg(-fgeg)}	&\gs{-fgeg<(-fgeg<)}		&\gs{-fgeg<v>\tld</v>(-fgeg~)}\\
\hline
Climacus vel virga subpunctis
				&\gs{gvFE(gvFE) hvGFE(hvGFE)}
								&\gs{gvFE>(gvFE>)}		&\gs{gvFE<v>\tld</v>(gvFE~)}\\
\hline
Scandicus vel virga præbipunctis
				&\gs{fh!iv(fh!iv) eghv(eghv)}
								&\gs{ghi>(ghi>)}		&\gs{fhi<v>\tld</v>(fhi~)}	\\
\hline
Salicus			&\gs{fgOh(fgOh)}	&\gs{fgOh>(fgOh>)}		&\gs{fgOh<v>\tld</v>(fgOh~)}\\
\hline
Trigonus			&\gs{GGF(GGF)}		&\gs{GGF>(GGF>)}		&\gs{GGF<v>\tld</v>(GGF~)}\\
\hline
Pressus			&\gs{ggof(ggof)}	&\gs{ggof>(ggof>)}		&\gs{ggof<v>\tld</v>(ggof~)}\\
\hline
Pressus minor		&=virga strata?		&					&\gs{ggo<v>\tld</v>(ggo~)}\\
\hline
Ancus?			&				&					&					\\
\hline
Pes subpunctis vel subbipunctis
				&\gs{ehF(ehF) fhGF(fhGF)}
								&\gs{ehGF>(ehGF>)}		&\gs{ehGF<v>\tld</v>(ehGF~)}\\
\hline
Pes stratus			&\gs{ghO(ghO) fhO(fhO)}
								&					&\\
\hline
Virga strata		&\gs{ggo(ggo)}		&					&\gs{ggo<v>\tld</v>(ggo~)}\\
\hline
Uncinus			& Laon à intégrer?	&					&\\
\hline
Scandicus flexus vel flexa præpunctis
				&\gs{FG@he(FG@he)}
								&\gs{FG@he>(FG@he>)}	&\gs{FG@he<v>\tld</v>(FG@he~)}\\
\hline
Scandicus subpunctis	&\gs{eghvFD(eghvFD)}
								&\gs{eghvFD<(eghvFD<)}	&\gs{eghvFD<v>\tld</v>(eghvFD~)}	\\
\hline


\end{longtable}

\end{document}

